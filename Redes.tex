\documentclass[12pt]{report} 

\newcommand{\changefontheaders}{\fontsize{8}{11}\selectfont}	
\usepackage{fancyhdr}
\fancyhf{}
\fancyhead[LE,RO]{\changefontheaders \slshape \rightmark}
\fancyfoot[C]{\thepage} 
\pagestyle{fancy}

\usepackage[spanish]{babel} 
\usepackage{amsmath,amsthm,amssymb,graphicx,verbatim}
\usepackage{dsfont,pifont} %for Numerical letters for numbers
\usepackage[svgnames]{xcolor}
\usepackage[backref,colorlinks=true,citecolor=black,linkcolor=black]{hyperref} 
\usepackage{pstricks}
\usepackage{lmodern} %to get bold in ttfamily
\usepackage{marginnote}
\usepackage{amsfonts}
\usepackage{amstext}
%\usepackage{tikz}
%\usetikzlibrary{shapes,arrows,automata,shadows,positioning,fit}
\usepackage{rotating}
\usepackage{multirow}
\usepackage[all]{xy}
\usepackage{xspace}
\usepackage{graphicx}
%\graphicspath{  {/home/jimena/Documentos/Tesis/} }
%\usepackage{itemsize}
%\usepackage[all]{xy}
%\usepackage{rotating}

%\renewcommand{\tablename}{Tabla} 
%\renewcommand{\listtablename}{\'Indice de Tablas}
%\newcommand{\ingles}[1]{\textit{#1}}
%\newcommand{\herramienta}[1]{\texttt{#1}}
%\newcommand{\codigo}[1]{\texttt{#1}}




\begin{document}

\section*{Redes}

Las redes tienen varios tama\~no, formas y figuras. Ni internet ni web son una red de computadoras. Internet no es una red \'unica, sino una red de redes, y web es un  sistema distribuido que se ejecuta sobre internet.



Las redes no tienen tama\~no ni forma. Internet no es una red \'unica, sino es una red de redes, y web es un sistema distribuido que se ejecuta a trav\'es de Internet.
La diferencia entre una red de computadoras y un sistema distribuido es:


Sistema distribuido: es un conjunto de computadoras independientes aparece ante 
sus usuarios como un sistema \'unico y consistente. Una capa de software que se ejecuta en el sistema operativo, es el responsable de implementar este modelo. Ejemplo de este sistema distribuido es WWW.     

Red de computadoras: la diferencia mayor con el sistema distribuido esta en el software (sobre todo en el sistema operativo), más que en el hardware. Lo que tienen en común los dos sistemas es que tienen que mover archivos.


Los sistemas están almacenados en computadoras que se llaman {\bf SERVIDORES}. Esos datos se encuentran alojados en una central y un administrador les da el mantenimiento.

\textit {Los empleados tienen m\'aquinas m\'as sencillas, llamadas {\bf CLIENTES},  pueden acceder a datos remotos}.

%\begin{figure}
%\includegraphics{aplicaciones de negocios en redes}

\begin{figure}[h!]
\includegraphics[scale=0.6]{images/cliente-servidor.eps}
\caption{Esquema}
\label{cliente-servidor}
\end{figure}

A esto se lo conoce como {\bf CLIENTE-SERVIDOR}, hay dos procesos involucrados, unos es la m\'aquina del cliente y el otro es la del servidor.

La comunicaci\'on toma la siguiente forma: el cliente env\'ia una solicitud a trav\'es de la red al servidor y espera una respuesta. Cuando el servidor recibe la solicitud, realiza el trabajo  que le pide o busca los datos solicitados y devuelve una repuesta. 

%\begin{figure}
%\includegraphics{cliente-servidor}
%\end{figure}

%\begin{figure}[h!]
%\includegraphics[scale=0.6]{images/cliente-servidor.eps}
%\caption{Esquema}
%\label{cliente-servidor}
%\end{figure}


\section*{Redes de Area Amplia}

Esta red (WAN), contiene un conjunto de m\'aquinas con programas de usuarios, estas maquinas se las llama \textit {HOST}. Los \textit {HOST} est\'an conectados por una \textit {\bf SUBRED}. Los clientes son los que poseen los host y los proveedores de Internet poseen y operan la subred.
  
En la mayor\'ia de las redes de \'area amplia la subred consta de dos componentes diferentes: 
\begin{enumerate}

\item \textit {L\'inea de transmisi\'on}: env\'ia bits entre maquinas. Pueden estar hechas de cable de cobre, fibra \'optica o, incluso, radio enlaces. 

\item \textit {Elementos de conmutaci\'on}: computadoras especializadas que conectan tres o m\'as l\'ineas de transmisi\'on.
\end{enumerate}

Cuando los datos llegan a una l\'inea de entrada, el elemento de conmutaci\'on debe elegir una l\'inea de salida para poder ser enviados. Estas computadoras de conmutaci\'on reciben varios nombres; conmutadores y enrutadores son los m\'as comunes.

En algunos casos el host puede estar conectado a una LAN en la que existe un enrutador, pero en algunos casos puede estar conectado de manera directa a un enrutador.  El conjunto de l\'ineas de comunicaci\'on y enrutadores forma la subred.

%\begin{figure}
%\includegraphics{redes de area amplia}
%\end{figure}


%\begin{figure}[!h]
%\includegraphics[scale=0.6]{images/modelChecking}
%\caption{Esquema b\'asico del proceso de model checking.}
%\label{modelchecking-IO}
%\end{figure}
 
 La mayor\'ia de las {\bf WANs}, la red contiene numerosas l\'ineas de transmisi\'on, cada una de las cuales conecta un par de enrutadores. Si dos enrutadores que no comparten una l\'inea de transmisi\'on quieren conectarse, lo deben hacer de manera indirecta a trav\'es de otros enrutadores.
  
Cuando un proceso de cualquier host tiene un mensaje que se va a enviar a un proceso de otro host, el host emisor divide primero el mensaje en paquetes, los cuales tienen un n\'umero de secuencia. Estos paquetes se env\'ian por la red uno a uno en una r\'apida sucesi\'on. Se transportan de manera individual a trav\'es de la red y se depositan en los host receptor, en donde se compactan en el mensaje original y se entregan al proceso receptor. 

\section*{Software de Redes}
\subsection*{Jerarqu\'ias de Protocolos}

Las mayor\'ias de las redes est\'an organizadas por una pila de \textit {CAPAS o NIVELES}, cada una construidas a partir de la que esta debajo de ella. 

La capa {\bf n} de una m\'aquina tiene comunicaci\'on con la capa {bf n} de otra m\'aquina. Un \underline {protocolo} es un acuerdo entre las partes sobre c\'omo se debe llevar a cabo la comunicaci\'on.
 
Las entidades que abarcan las capas correspondientes en diferentes m\'aquinas se llaman \textit {IGUALES}. Los iguales son los que se comunican a trav\'es del protocolo.

%\begin{figure}
%\includegraphics{capas de protocolo}
%\end{figure}

Los datos no se trasfieren de la capa {\bf n}  a la otra capa {\bf n} de manera directa, sino que cada capa pasa los datos e informaci\'on a la capa inmediatamente inferior, hasta que se alcanza la capa m\'as baja. Debajo de la capa uno se encuentra el {\bf Medio F\'isico} a trav\'es del cual ocurre la comunicaci\'on real.

La comunicaci\'on virtual se muestra con l\'ineas punteadas, la f\'isica, con l\'ineas s\'olidas.

Entre cada par de capas adyacentes est\'a una \textit {Interfaz}, define que operaciones y servicios primitivos pone la capa m\'as baja a disposici\'on de la capa superior.

(Ver  m\'as detalles en la página 27 y 28 de Tanenbaum)

\section*{Relaci\'on de Servicios a Protocolos}

Servicios y protocolos son conceptos distintos. Un \underline {Servicio} es un conjunto de operaciones que una capa proporciona a la capa que est\'a sobre ella. Un servicio esta relacionado con la interfaz entre dos capas, donde la capa inferior es la que provee el servicio y la superior es la que recibe. 

Un \underline {Protocolo} es un conjunto de reglas que rigen el formato y el significado de los paquetes, o mensajes. El servicio y el protocolo no depende uno del otro.
(Ver  m\'as detalles en la p\'agina 36 y 37 de Tanenbaum).

\section*{Modelos de Referencias}
\subsection*{El modelo de Referencias TCP/IP}

Cuando se agregaron redes satelitales y de radio, los protocolos existentes tuvieron problemas, por lo que se necesito una nueva arquitectura de referencia. La capacidad para conectar m\'ultiples redes fue una de las principales metas de dise\~no desde sus inicios. M\'as tarde, est\'a arquitectura se lleg\'o a conocer como el \textit {\bf Modelo de referencias TCP/IP}.

\subsection*{La Capa de Interred}

Todos los requerimientos se basaron en la elecci\'on de una red de conmutaci\'on de paquetes basadas en una nueva capa de interred no orientada a la conexi\'on. Esta capa, llamada {\bf Capa de Interred}, es la clave que mantiene unida a la arquitectura. Su trabajo es permitir que los HOST inyecten paquetes dentro de cualquier red y que viajen a su destino de manera independiente.
 
La capa de interred define un paquete de formato y protocolo llamado IP (Protocolo de Internet). Por lo que el trabajo de esta capa es entregar paquetes IP al destinatario.

\subsection*{La Capa de Transporte}

La capa que est\'a arriba de la capa de interred en el modelo de TCP/IP se la llama  {\bf Capa de Transporte}. Esta dise\~na para que las entidades iguales en los hosts de origen y destino puedan llevar a cabo una conversaci\'on.

El primero, TCP (Protocolo de Control de Transmisión) es confiable orientado a la conexi\'on, permite que un conjunto de bytes que se originan en una m\'aquina se entregue sin errores en cualquier otra m\'aquina en la interred. El proceso TCP receptor re ensambla en el flujo de salida los mensajes recibidos. TCP maneja el control de flujo para asegurarse de que un emisor r\'apido no sature a un receptor lento con m\'as mensajes de los que puede manejar. 

%\begin{figure}
%\includegraphics{capas de transporte}
%\end{figure}

El segundo protocolo de esta capa, UDP (Protocolo de Datagrama de Usuario), no confiable y no orientado a la conexi\'on para aplicaciones que no desean el control de flujo de TCP y que desean proporcionar el suyo. 
Tambi\'en tiene un amplio uso de consultas \'unicas de solicitud-repuesta de tipo Cliente-Servidor en un solo env\'io. 

\subsection*{La Capa de Aplicaci\'on}  
 

El modelo TCP/IP no tiene capa de sesi\'on ni de representaci\'on.
 
Arriba de la capa de transporte est\'a la {\bf Capa de Aplicaci\'on}, contiene todos los protocolos de nivel m\'as alto. Los primeros incluyeron una terminal virtual (TELNET), transferencia de archivos (FTP) y correo electr\'onico (SMTP). El protocolo virtual de terminal virtual permite que un usuario en una m\'aquina se registre en una m\'aquina remota y trabaje ah\'i.
 
El protocolo de transferencia de archivos proporciona de manera muy eficiente de mover archivos de una m\'aquina a otra. El correo electr\'onico era originalmente solo un tipo de transferencia de archivos, pero m\'as tarde se desarroll\'o un protocolo especializado (SMTP). Con el tiempo se han agregado muchos otros protocolos: DNS (Sistemas de Nombres de Dominios)resoluciones de nombres de hosts en sus direcciones de red; NNTP, para trasportar los art\'iculos de USENET; HTTP, para las p\'aginas de WWW. 

\subsection*{Uso de Internet}

Una m\'aquina est\'a en internet si ejecuta la pila de protocolos de TCP/IP, tiene una direcci\'on IP y puede enviar paquetes IP a todas las dem\'as m\'aquinas en internet. 
Internet y sus predecesores ten\'ian cuatro aplicaciones principales: 
%\begin{enumerate}
%Correo Electr\'onico: la capacidad para  redactar, enviar y recibir correo electr\'onico ha sido posible desde los inicios de ARPANET y su gran popularidad.

%Noticias: los grupos de noticias son foros especializados en los que los usuarios con un inter\'es com\'un pueden intercambiar mensajes.

%Inicio Remoto de Sesi\'on: los usuarios de cualquier parte pueden iniciar sesi\'on en cualquier otra m\'aquina que tenga cuenta.

%Transferencias de Archivos: con el programa de FTP, los usuarios pueden copiar archivos en Internet a otra. WWW hizo posible que un sitio estableciera  p\'aginas de informaci\'on que contienen texto, im\'agenes, sonidos e incluso v\'ideo, y v\'inculos integrados   a otras p\'aginas.
%s\end{enumerate}


\subsection*{Arquitectura de Internet}

Un cliente en casa llama a su ISP (empresas llamadas proveedores de servidores de internet) desde una l\'inea telef\'onica conmutada. El modem es una tarjeta dentro de su PC que convierte las se\~nales digitales que la computadora produce en se\~nales analogas que pueden pasar sin obst\'aculos a trav\'es del sistema telef\'onico. Estas se\~nales se trasfieren al POP (punto de Presencia) del ISP, donde se retiran del sistema telef\'onico y se inyectan en la red regional del ISP. El sistema es totalmente digital y de conmutación de paquetes. Si el ISP es la telco local, es probable que el POP est\'a ubicado en el conmutador telef\'onico, donde termina el cableado de tel\'efono de los clientes.   

(Ver  m\'as detalles en la p\'agina 58 y 59 de Tanenbaum)
%\begin{figure}
%\includegraphics{Arquitectura de Internet}
%\end{figure}

\section*{Capa de Transporte}

\subsection*{Servicio proporcionados a las capas superiores}		

La meta principal de esta capa es brindar servicio eficiente, confiable y econ\'omico a sus usuarios. Para lograr este objetivo, la capa de transporte utiliza servicios proporcionados por la capa de red. 

El servicio de transporte orientado a la conexi\'on es parecido al servicio de red orientado a la conexi\'on. las conexiones tienen tres fases: establecimiento, transferencia de datos y liberaci\'on (o terminaci\'on). El c\'odigo de transporte se ejecuta en la m\'aquina del usuario, pero la capa de red, se ejecuta en los erutadores, los cuales son operadores por la empresa portadora.

Los usuarios no tienen control sobre la capa de red, por lo que no pueden resolver el problema del mal servicio. La \'unica posibilidad es poner otra capa por encima de la capa de red otra capa que mejore la calidad del servicio. 

La capa de transporte puede detectar y compensar paquetes perdidos y datos alterados.

\subsection*{Primitivas del Servicio de Transporte}

El servicio de transporte es parecido al servicio de red, hay algunas diferencias importantes. La principal es, el servicio de red es modelar el servicio por las redes reales. Estas pueden perder paquetes, por lo que el servicio de red no es confiable. 

El servicio de transporte (orientado a la conexi\'on) s\'i es confiable. las redes reales no est\'an libres de errores, \'ese es el prop\'osito de la capa de transporte: ofrecer un servicio confiable en una red no confiable.


        \section*{?`Qu\'e es Maven?}

Maven es una herramienta de software para la gesti\'on y construcci\'on de proyectos Java creada por Jason van Zyl, de Sonatype, en 2002.
Maven utiliza un Project Object Model (POM) para describir el proyecto de software a construir, sus dependencias de otros m\'odulos y componentes externos, y el orden de construcci\'on de los elementos. Viene con objetivos predefinidos para realizar ciertas tareas claramente definidas, como la compilaci\'on del c\'odigo y su empaquetado.

Una caracter\'istica clave de Maven es que est\'a listo para usar en red. El motor incluido en su n\'ucleo puede din\'amicamente descargar plugins de un repositorio, el mismo repositorio que provee acceso a muchas versiones de diferentes proyectos Open Source en Java, de Apache y otras organizaciones y desarrolladores.

\subsection*{Ciclo de Vida}

Las partes del ciclo de vida principal del proyecto Maven son:

    {\bf compile}: Genera los ficheros .class compilando los fuentes .java
    {\bf test}: Ejecuta los test autom\'aticos de JUnit existentes, abortando el proceso si alguno de ellos falla.
    {\bf package}: Genera el fichero .jar con los .class compilados
    {\bf install}: Copia el fichero .jar a un directorio de nuestro ordenador donde maven deja todos los .jar. De esta forma esos .jar pueden utilizarse en otros proyectos maven en el mismo ordenador.
    {\bf deploy}: Copia el fichero .jar a un servidor remoto, poni\'endolo disponible para cualquier proyecto maven con acceso a ese servidor remoto.

Cuando se ejecuta cualquiera de los comandos maven, por ejemplo, si ejecutamos mvn install, maven ir\'a verificando todas las fases del ciclo de vida desde la primera hasta la del comando, ejecutando solo aquellas que no se hayan ejecutado previamente.




\end{document}
